\documentclass{VUMIFPSbakalaurinis}
\usepackage{algorithmicx}
\usepackage{algorithm}
\usepackage{algpseudocode}
\usepackage{amsfonts}
\usepackage{amsmath}
\usepackage{bm}
\usepackage{caption}
\usepackage{color}
\usepackage{float}
\usepackage{graphicx}
\usepackage{listings}
\usepackage{subfig}
\usepackage{wrapfig}

\usepackage{enumitem}
\setitemize{noitemsep,topsep=0pt,parsep=0pt,partopsep=0pt}
\setenumerate{noitemsep,topsep=0pt,parsep=0pt,partopsep=0pt}

% Ignore all trivial warnings
\hbadness=5000
% Titulinio aprašas
\university{Vilniaus universitetas}
\faculty{Matematikos ir informatikos fakultetas}
\department{Programų sistemų katedra}
\papertype{Bakalauro baigiamojo darbo planas}
\title{Srautinio apdorojimo modulių generavimas kintant rodiklių duomenų struktūrai}
\titleineng{Generation of Stream Processing Modules upon Change of Indicator Data Structure}
\author{Vytautas Žilinas}
\supervisor{lekt. Andrius Adamonis}
\date{Vilnius – \the\year}

% Nustatymai
% \setmainfont{Palemonas}   % Pakeisti teksto šriftą į Palemonas (turi būti įdiegtas sistemoje)
\bibliography{bibliografija}

\begin{document} 
\maketitle

\cleardoublepage\pagenumbering{arabic}
\setcounter{page}{2}

\section{Tyrimo objektas ir aktualumas}
Šiame darbe nagrinėjami rodiklių duomenų apdorojimas. Rodiklių duomenimis vadiname duomenis, aprašančius kažkokių objektų savybes arba veiklos procesų rezultatus, 
duomenis galima transformuoti, analizuoti ir grupuoti pagal pasirinktus rodiklius, 
pavyzdžiui: bazinė mėnesinė alga, mirusiųjų skaičius pagal mirties priežastis, krituliai per metus. Taip pat rodiklių struktūra gali keistis laikui bėgant: 
objektų atributų taksonomija (pvz. mirties priežasčių sąrašas, finansinių sąskaitų sąrašas) arba įrašo atributų sąrašai.
Rodiklių duomenų bazės pasižymi tuo, kad duomenys į jas patenka iš daug skirtingų tiekėjų ir patekimo laikas tarp tiekėjų nėra 
sinchronizuojamas, o suagreguotą informaciją vartotojai gali užklausti bet kurio metu.  \par
Vienas iš būdų apdoroti didelius kiekius duomenų gyvai (angl. Real-time data processing) yra srautinis duomenų apdorojimas \cite{BigData, StreamProcessingInData}. 
Dabartinės srautinio apdorojimo sistemos leidžia kurti modulius, kurie apdoroja duomenis ir talpina juos į atskirą talpyklą \cite{8Requirements}. 
Šiame darbe kuriamas sprendimas yra aktualus, kai kinta duomenų struktūra ir norėtume šis sprendimas prisitaiko 
prie duomenų pokyčių kurdamas naujus apdorojimo modulius.

\subsection{Darbo tikslas}
Sukurti architektūrą, kuris dinamiškai prisitaiko prie rodiklių duomenų struktūrų pokyčių, remiantis srautinio apdorojimo platforma.

\subsection{Uždaviniai}
\begin{enumerate}
    \item Apibrėžti rodiklių duomenų modelį ir galimus rodiklių duomenų struktūros pokyčius.
    \item Nustatyti, kaip specifikuoti duomenų struktūrą ir kaip atrodys aprašymas, kai keisis rodiklių duomenų struktūra.
    \item Pasirinkti srautinio duomenų apdorojimo sistemą, joje sukurti sudarytos architektūros sprendimą ir atlikti bandymus.
\end{enumerate}

\subsection{Laukiami rezultatai}
\begin{enumerate}
    \item Apibrėžta rodiklių duomenų struktūrą ir galimi duomenų struktūros pokyčiai.
    \item Su pasirinkta srautinio apdorojimo sistema sukurtas srautinio apdorojimo programų generavimo sprendimas ir eksperimentui reikalingos pagalbinės programos. 
\end{enumerate}
% Tikimasi, kad pasiūlytas sprendimas bus naudingas dažnai nežymiai kintant duomenų struktūrai. Taip pat pasirinkta srautinio 
% duomenų apdorojimo sistema leis nesudėtingai įgyvendinti modulių kūrimą.
\section{Tyrimo metodas ir numatomas darbo atlikimo procesas}

\subsection{Tyrimo metodas}
Numatomas tyrimo metodas: eksperimentas, kurio metu bus aprašytos duomenų struktūros ir duomenų struktūrų pokyčiai. Remiantis interneto straipsniais ir srautinio apdorojimo sistemų dokumentacijomis bus pasirinkta srautinio apdorojimo sistema sprendimui kurti. Bus sukurtas srautinio apdorojimo modulių generatorius (remiantis knyga: Code Generation in Action parašyta Jack Herrington) ir aprašytas sprendimo generuojamas kodas. \par 

\subsection{Darbo atlikimo procesas}
\begin{enumerate}
    \item Apibrėžtas rodiklių duomenų modelis, XML arba JSON formatu keliaujantis duomenys.
    \item Apibrėžtas būdas stebėti duomenų pokyčius, lyginimas įeinančių duomenų su schema arba kitaip aprašyta struktūra.
    \item Išanalizuotos esamos srautinio apdorojimo sistemos ir pasirinkta tinkamiausia sistemą sprendimui pagal:
    	\begin{itemize}
		\item Galimybę pridėti modulį.
		\item Srautinio apdorojimo sistemos perkrovimo paprastumą.
	\end{itemize}
    \item Su pasirinkta sistema sukurtas sprendimas ir pagalbinės programos eksperimentui.
\end{enumerate}

% \section{Darbui aktualus literatūros šaltiniai}
% Aktualu: https://github.com/manuzhang/awesome-streaming \\
% https://recruitloop.com/blog/rise-big-data-data-science-infographic/ \\
% https://wso2.com/library/articles/2018/02/stream-processing-101-from-sql-to-streaming-sql-in-ten-minutes/ \\
% https://dataconomy.com/2017/03/care-big-data-care-stream-processing/ \\
% Possible frameworks: \\
%  Python - https://github.com/robinhood/faust \\
%  Python - https://github.com/WallarooLabs/wallaroo \\
%  Storm with python - https://github.com/Parsely/streamparse \\
%  Maybe just KSQL - https://github.com/confluentinc/ksql \\
 
\printbibliography[heading=bibintoc] 

\end{document}
