\documentclass{VUMIFPSbakalaurinis}
\usepackage{algorithmicx}
\usepackage{algorithm}
\usepackage{algpseudocode}
\usepackage{amsfonts}
\usepackage{amsmath}
\usepackage{bm}
\usepackage{caption}
\usepackage{color}
\usepackage{float}
\usepackage{graphicx}
\usepackage{listings}
\usepackage{subfig}
\usepackage{wrapfig}

\usepackage{enumitem}
\setitemize{noitemsep,topsep=0pt,parsep=0pt,partopsep=0pt}
\setenumerate{noitemsep,topsep=0pt,parsep=0pt,partopsep=0pt}

% Ignore all trivial warnings
\hbadness=5000
% Titulinio aprašas
\university{Vilniaus universitetas}
\faculty{Matematikos ir informatikos fakultetas}
\department{Programų sistemų katedra}
\papertype{Bakalauro baigiamojo darbo planas}
\title{Srautinio apdorojimo modulių generavimas kintant rodiklinių duomenų struktūrai}
\titleineng{Generation of Stream Processing Modules upon Change of Indicator Data Structure}
\author{Vytautas Žilinas}
\supervisor{lekt. Andrius Adamonis}
\date{Vilnius – \the\year}

% Nustatymai
% \setmainfont{Palemonas}   % Pakeisti teksto šriftą į Palemonas (turi būti įdiegtas sistemoje)
\bibliography{bibliografija}

\begin{document} 
\maketitle

\cleardoublepage\pagenumbering{arabic}
\setcounter{page}{2}

\section{Tyrimo objektas ir aktualumas}
Šiame darbe nagrinėjami rodiklių duomenys. Šiuos duomenis galima transformuoti, analizuoti ir grupuoti pagal pasirinktus rodiklius, 
pavyzdžiui: bazinė mėnesinė alga, mirusiųjų skaičius pagal mirties priežastis, krituliai per metus. 
Rodiklių duomenų bazės pasižymi tuo, kad duomenys į jas patenka iš daug skirtingų tiekėjų ir patekimo laikas tarp tiekėjų nėra 
sinchronizuojamas, o suagreguotą informaciją vartotojai gali užklausti bet kurio metu.  \par
Šiuo metu pasaulyje vis didėja duomenų kiekis, kuriuos gali apdoroti kompiuteriai \cite{DataTrend}. 
Dėl šios priežasties vis labiau paklausus tampa duomenų apdorojimo specialistai \cite{IBMprediction}. 
Vienas iš būdų apdoroti didelius kiekius duomenų gyvai (angl. Real-time data processing) yra srautinis duomenų apdorojimas \cite{BigData, StreamProcessingInData}. 
Dabartinės srautinio apdorojimo architektūros leidžia kurti modulius, kurie apdoroja duomenis ir talpina juos į atskirą talpyklą \cite{8Requirements}. 
Šiame darbe kuriamas sprendimas yra aktualus, kai dažnai kinta duomenų struktūra, kadangi šis sprendimas prisitaiko 
prie duomenų pokyčių kurdamas naujus apdorojimo modulius.

\subsection{Darbo tikslas}
Darbo tikslas: Sukurti bandomąjį rodiklinių duomenų srautinio apdorojimo modulių 
generavimo sistemos modelį ir eksperimento būdu palyginti jo tinkamumą apibrėžtam 
uždaviniui spręsti su rankiniu srautinio apdorojimo modulių rašymu.

\subsection{Uždaviniai}
Uždaviniai:
\begin{enumerate}
    \item Atlikti skirtingų srautinio duomenų apdorojimo sprendimo architektūrų analizę ir pasirinkti vieną iš jų eksperimentui.
    \item Sukurti testinių duomenų generatorių.
    \item Realizuoti dinaminį srautinio duomenų apdorojimo architektūros modulių kūrimo sprendimą.
    \item Palyginti sukurtą sprendimą su rankiniu srautinio apdorojimo modulių rašymu.
\end{enumerate}

\subsection{Laukiami rezultatai}
    \begin{enumerate}
        \item Sukurtas testinių duomenų generatorius ir sukurtas pasirinktos srautinio apdorojimo architektūros sprendimas.
        \item Išmatuota sukurto sprendimo greitaveiką.
        \item Padaryta sprendimo vertės ir kainos analize ir palyginta su rankiniu srautinių duomenų apdorojimo moduliu kūrimo sprendimu. 
    \end{enumerate}
    \vspace{1 mm}
% Tikimasi, kad pasiūlytas sprendimas bus naudingas dažnai nežymiai kintant duomenų struktūrai. Taip pat pasirinkta srautinio 
% duomenų apdorojimo architektūra leis nesudėtingai įgyvendinti modulių kūrimą.
\section{Tyrimo metodas ir numatomas darbo atlikimo procesas}
Numatomas eksperimentinis tyrimo metodas.
Darbo atlikimo procesas:
\begin{enumerate}
    \item Išanalizuotos esamos srautinio apdorojimo architektūros ir pasirinkta tinkamiausia architektūra sprendimui pagal:
    	\begin{itemize}
		\item Galimybę pridėti modulį.
		\item Srautinio apdorojimo sistemos perkrovimo paprastumą.
		\item Modulių kūrimo sudėtingumą.
		\item Dokumentaciją.
	\end{itemize}
    \item Sukurtas testinių duomenų generatorius, kuris periodiškai keičia duomenų struktūrą (prideda ir šalina laukus) .
    \item Su pasirinkta architektūra sukurtas sprendimas eksperimentui ir fiksuojamas laikas ir atliktos užduotis.
    \item Pagal surinktus duomenis atliekama analizė, kuri parodo ar verta kurti dinamišką srautinio duomenų apdorojimo sprendimą 
    ar naudoti rankinę.  
\end{enumerate}

% \section{Darbui aktualus literatūros šaltiniai}
% Aktualu: https://github.com/manuzhang/awesome-streaming \\
% https://recruitloop.com/blog/rise-big-data-data-science-infographic/ \\
% https://wso2.com/library/articles/2018/02/stream-processing-101-from-sql-to-streaming-sql-in-ten-minutes/ \\
% https://dataconomy.com/2017/03/care-big-data-care-stream-processing/ \\
% Possible frameworks: \\
%  Python - https://github.com/robinhood/faust \\
%  Python - https://github.com/WallarooLabs/wallaroo \\
%  Storm with python - https://github.com/Parsely/streamparse \\
%  Maybe just KSQL - https://github.com/confluentinc/ksql \\
 
\printbibliography[heading=bibintoc] 

\end{document}
