\documentclass{VUMIFPSbakalaurinis}
\usepackage{algorithmicx}
\usepackage{algorithm}
\usepackage{algpseudocode}
\usepackage{amsfonts}
\usepackage{amsmath}
\usepackage{bm}
\usepackage{caption}
\usepackage{color}
\usepackage{float}
\usepackage{graphicx}
\usepackage{listings}
\usepackage{subfig}
\usepackage{wrapfig}

\usepackage{enumitem}
%PAKEISTA, tarpai tarp sąrašo elementų
\setitemize{noitemsep,topsep=0pt,parsep=0pt,partopsep=0pt}
\setenumerate{noitemsep,topsep=0pt,parsep=0pt,partopsep=0pt}


% Titulinio aprašas
\university{Vilniaus universitetas}
\faculty{Matematikos ir informatikos fakultetas}
\department{Programų sistemų katedra}
\papertype{Bakalauro baigiamojo darbo planas}
\title{Srautinio apdorojimo modulių generavimas kintant rodiklinių duomenų struktūrai}
\titleineng{Generation of Stream Processing Modules upon Change of Indicator Data Structure}
\author{Vytautas Žilinas}
\supervisor{lekt. Andrius Adamonis}
% \reviewer{doc. dr. Vardauskas Pavardauskas}
\date{Vilnius – \the\year}

% Nustatymai
% \setmainfont{Palemonas}   % Pakeisti teksto šriftą į Palemonas (turi būti įdiegtas sistemoje)
% \bibliography{bibliografija}

\begin{document} 
\maketitle

\cleardoublepage\pagenumbering{arabic}
\setcounter{page}{2}


\section{Tyrimo objektas ir aktualumas}
Įvade nurodomas darbo tikslas ir uždaviniai, kuriais bus įgyvendinamas tikslas,
aprašomas temos aktualumas, apibrėžiamas tiriamasis objektas akcentuojant
neapibrėžtumą, kuris bus išspręstas darbe, aptariamos teorinės darbo prielaidos
bei metodika, apibūdinami su tema susiję literatūros ar kitokie šaltiniai,
temos analizės tvarka, darbo atlikimo aplinkybės, pateikiama žinių apie
naudojamus instrumentus (programas ir kt., jei darbe yra eksperimentinė dalis).
Darbo įvadas neturi būti dėstymo santrauka. Įvado apimtis 2–4 puslapiai.

\section{Darbo tikslas}
Darbo tikslas: Sukurti bandomąjį rodiklinių duomenų srautinio apdorojimo modulių 
generavimo sistemos modelį ir eksperimento būdu palyginti jo tinkamumą apibrėžtam 
uždaviniui spręsti su esamais sprendimo būdais.

\section{Uždaviniai}
Uždaviniai:
\begin{enumerate}
  \item Atlikti skirtingų srautinio duomenų apdorojimo sprendimo architektūrų analizę ir pasirinkti vieną iš jų ekperimentui.
  \item Sukurti testinių duomenų generatorių.
  \item Realizuoti dinaminį srautinio duomenų apdorojimo architektūros modulių kūrimo sprendimą.
	\item Palyginti sukurtą sprendimą su alternatyvomis.
\end{enumerate}

\section{Laukiami rezultatai}
    \begin{enumerate}
        \item Sukurtas testinių duomenų generatorius ir sukurtas pasirinktos srautinio apdorojimo architekturos sprendimas.
        \item Išmatuota sukurto sprendimo greitaveiką.
        \item Padaryta sprendimo vertės ir kainos analize ir palyginta su rankiniu srautinių duomenų apdorojimo moduliu kūrimo sprendimu. 
    \end{enumerate}
    \vspace{1 mm}
% Tikamasi, kad pasiūlytas sprendimas bus naudingas dažnai nežymiai kintant duomenų strūkturai. Taip pat pasirinkta srautinio 
duomenų apdorojimo architektūra leis nesudetingai įgyvendinti modulių kūrimą.
\section{Tyrimo metodas ir numatomas darbo atlikimo procesas}
Medžiagos darbo tema dėstymo skyriuose išsamiai pateikiamos nagrinėjamos temos
detalės: pradiniai duomenys, jų analizės ir apdorojimo metodai, sprendimų
įgyvendinimas, gautų rezultatų apibendrinimas.

Medžiaga turi būti dėstoma aiškiai, pateikiant argumentus. Tekste dėstomas
trečiuoju asmeniu, t.y. rašoma ne „aš manau“, bet „autorius mano“, „autoriaus
nuomone“. Reikėtų vengti informacijos nesuteikiančių frazių, pvz., „...kaip jau
buvo minėta...“, „...kaip visiems žinoma...“ ir pan., vengti grožinės
literatūros ar publicistinio stiliaus, gausių metaforų ar panašių meninės
išraiškos priemonių.

Skyriai gali turėti poskyrius ir smulkesnes sudėtines dalis, kaip punktus ir
papunkčius.

\section{Darbui aktualus literatūros šaltiniai}
Aktualu: https://github.com/manuzhang/awesome-streaming \\
Possible frameworks: \\
 Python - https://github.com/robinhood/faust \\
 Python - https://github.com/WallarooLabs/wallaroo \\
 Storm with python - https://github.com/Parsely/streamparse \\
 Maybe just KSQL - https://github.com/confluentinc/ksql \\
Medžiagos darbo tema dėstymo skyriuose išsamiai pateikiamos nagrinėjamos temos
detalės: pradiniai duomenys, jų analizės ir apdorojimo metodai, sprendimų
įgyvendinimas, gautų rezultatų apibendrinimas.

\end{document}
